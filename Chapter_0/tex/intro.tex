\documentclass{article}
\usepackage{amsmath}
\usepackage{xeCJK}
\usepackage[
    a4paper,
    left=2.54cm,
    right=2.54cm,
    top=3.18cm,
    bottom=3.18cm,
    includehead  % 包含页眉在边距内
]{geometry}
\setCJKmainfont{SimSun}

\title{程序 = 数据结构 + 算法}
\author{110gituser}
\date{2025}

\begin{document}

\maketitle

\section*{“程序 = 数据结构 + 算法”现实应用举例}

这是计算机科学家尼古拉斯·沃斯(Niklaus Wirth)提出的公式,强调程序设计的两个核心要素:\textbf{数据结构(存储信息的方式)}和\textbf{算法(处理信息的方法)}。以下是几个现实生活中的应用示例:

\subsection*{例子一:导航软件(如高德地图、百度地图)}

\begin{itemize}
    \item \textbf{应用背景}:你想从 A 地导航到 B 地。
    \item \textbf{数据结构}:城市路网模型,通常用图(Graph)来表示,交叉路口是节点,道路是边。
    \item \textbf{算法}:路径搜索算法,比如 A* 算法、Dijkstra 算法,用于找到最短或最快的路径。
    \item \textbf{总结}:\textbf{地图程序 = 路网图(数据结构) + 路径搜索算法(算法)}
\end{itemize}

\subsection*{例子二:淘宝或京东的商品搜索}

\begin{itemize}
    \item \textbf{应用背景}:用户搜索“蓝牙耳机”。
    \item \textbf{数据结构}:倒排索引(Inverted Index)、哈希表、Trie 树等,用于存储关键词与商品的对应关系。
    \item \textbf{算法}:搜索算法 + 排序算法(如 BM25、TF-IDF、学习排序),用于根据相关度、销量等对结果排序。
    \item \textbf{总结}:\textbf{搜索系统 = 商品索引结构(数据结构) + 搜索与排序算法(算法)}
\end{itemize}

\subsection*{例子三:微信消息发送}

\begin{itemize}
    \item \textbf{应用背景}:你发一条消息给你的朋友。
    \item \textbf{数据结构}:消息队列、用户哈希表、聊天记录数据库(可能是键值对结构)。
    \item \textbf{算法}:消息投递算法、去重算法、同步算法等。
    \item \textbf{总结}:\textbf{即时通信程序 = 用户与消息的数据结构 + 实时传输与同步算法}
\end{itemize}

\subsection*{例子四:支付宝的风控系统}

\begin{itemize}
    \item \textbf{应用背景}:用户发起一次异常交易,系统需要判断是否拦截。
    \item \textbf{数据结构}:用户行为日志、交易记录(图结构或表结构)、模型特征向量。
    \item \textbf{算法}:机器学习算法(如决策树、XGBoost)、图分析算法,用于识别是否为欺诈行为。
    \item \textbf{总结}:\textbf{风控系统 = 用户交易数据结构 + 风险识别算法}
\end{itemize}

\subsection*{例子五:抖音的视频推荐系统}

\begin{itemize}
    \item \textbf{应用背景}:你打开抖音,系统给你推荐视频。
    \item \textbf{数据结构}:用户画像、视频特征、用户行为日志(矩阵、图结构)。
    \item \textbf{算法}:协同过滤、深度学习推荐算法(如神经网络),用于个性化推荐。
    \item \textbf{总结}:\textbf{推荐系统 = 用户与内容数据结构 + 推荐算法}
\end{itemize}

\end{document}